% !TEX TS-program = pdflatex
% !TEX encoding = UTF-8 Unicode

\documentclass[12pt]{article} 

\usepackage[utf8]{inputenc} 


%%% PAGE DIMENSIONS
\usepackage[margin=3.3cm]{geometry} 
\geometry{a4paper} 
\usepackage[parfill]{parskip} % Activate to begin paragraphs with an empty line rather than an indent

%%% PACKAGES
\usepackage{graphicx} % support the \includegraphics command and options
\usepackage{booktabs} % for much better looking tables
\usepackage{array} % for better arrays (eg matrices) in maths
\usepackage{paralist} % very flexible & customisable lists (eg. enumerate/itemize, etc.)
\usepackage{verbatim} % adds environment for commenting out blocks of text & for better verbatim
\usepackage{mathtools}
\usepackage[hidelinks]{hyperref}
\usepackage{dirtytalk}
\usepackage[usenames, dvipsnames]{color}
\urlstyle{same}
\usepackage{float}
\usepackage{caption}
\usepackage{subcaption}
\usepackage{booktabs}
\usepackage{setspace}
\usepackage{eurosym}
\usepackage{listings} %% for R Code
\usepackage{nomencl}
\usepackage{bm}
\usepackage{fancyhdr}
\usepackage{rotating}
\usepackage{graphicx}
\usepackage{subcaption}
\usepackage{amsfonts}
\usepackage{mathtools,bbm}%Use your favorite font for bboard number
%\usepackage[
%backend=biber,
%style=authoryear-icomp,
%citestyle=authoryear-icomp,
%maxcitenames=2
%]{biblatex}
%#\addbibresource{biblio.bib}

\usepackage{multirow}
\usepackage[dvipsnames]{xcolor}
\usepackage[toc,page]{appendix}
\usepackage{color, colortbl}
\definecolor{LightCyan}{HTML}{f2f2f2}
\usepackage{longtable}

\usepackage[round]{natbib}
\bibliographystyle{apa-good}
\renewcommand{\bibname}{References}


\makeatletter
\newcommand*\bigcdot{\mathpalette\bigcdot@{.5}}
\newcommand*\bigcdot@[2]{\mathbin{\vcenter{\hbox{\scalebox{#2}{$\m@th#1\bullet$}}}}}
\makeatother

\newcommand{\blue}[1]{\textcolor{blue}{#1}}

% ------------------------------------------
% Tuning of the NOMENCLATURE!
%% This code creates the groups
% -----------------------------------------
\usepackage{etoolbox}
\renewcommand\nomgroup[1]{%
  \item[\bfseries
  \ifstrequal{#1}{A}{Sets \& Indices}{%
  \ifstrequal{#1}{B}{Distributions}{%
  \ifstrequal{#1}{C}{Variables \& Arrays}{%
  \ifstrequal{#1}{E}{Other Symbols}{%
  \ifstrequal{#1}{D}{Functions}{%
  \ifstrequal{#1}{F}{Acronyms}{}}}}}}%
]}


\renewcommand{\nomlabel}[1]{#1 \dotfill}

\renewcommand{\appendixname}{Appendix: Full Simulation Results}
\renewcommand{\appendixtocname}{Appendix: Full Simulation Results}
\renewcommand{\appendixpagename}{\Large Appendix: Full Simulation Results}

% -----------------------------------------
%%%%%%%%%%%
%%%%%%%%%%%

%%% Listings options
\lstdefinelanguage{Renhanced}[]{R}{
  otherkeywords={!,!=,~,\$,*,\&,\%/\%,\%*\%,\%\%,<-,<<-, ::},
  morekeywords={},
  deletekeywords={hist, runif, plot,  R, read.table, read, check, text, file, attributes, quote, missing, c, any, which, na, deparse, structure, install, page, model, par, data, round, D, system, terms},
  alsoletter={.\%},%
  alsoother={:_\$}}
 \lstset{ 
  language=Renhanced,                     % the language of the code
  basicstyle=\small\ttfamily, % the size of the fonts that are used for the code
  numbers=left,                   % where to put the line-numbers
  numberstyle=\tiny\color{Blue},  % the style that is used for the line-numbers
  stepnumber=1,                   % the step between two line-numbers. If it is 1, each line will be numbered
  numbersep=10pt,                  % how far the line-numbers are from the code
  backgroundcolor=\color{white},  % choose the background color. You must add \usepackage{color}
  showspaces=false,               % show spaces adding particular underscores
  showstringspaces=false,         % underline spaces within strings
  showtabs=false,                 % show tabs within strings adding particular underscores
  frame=false,                   % adds a frame around the code
  rulecolor=\color{black},        % if not set, the frame-color may be changed on line-breaks within not-black text (e.g. commens (green here))
  tabsize=2,                      % sets default tabsize to 2 spaces
  captionpos=b,                   % sets the caption-position to bottom
  breaklines=true,                % sets automatic line breaking
  breakatwhitespace=false,        % sets if automatic breaks should only happen at whitespace
  keywordstyle=\color{RoyalBlue},      % keyword style
  commentstyle=\color{YellowGreen},   % comment style
  stringstyle=\color{ForestGreen}      % string literal style
} 
%\renewcommand{\lstlistingname}{Code-Chunk}
%%% New commands
\newcommand{\li}{\lstinline}
%%% HEADERS & FOOTERS
\usepackage{fancyhdr} % This should be set AFTER setting up the page geometry
\pagestyle{fancy} % options: empty , plain , fancy
\renewcommand{\headrulewidth}{0pt} % customise the layout...
\lhead{}\chead{}\rhead{}
\lfoot{}\cfoot{\thepage}\rfoot{}


%%%%----------- SECTION TITLE APPEARANCE -----------------------------
%%% 
%\usepackage{sectsty}
%\allsectionsfont{\sffamily\mdseries\upshape} % (See the fntguide.pdf for font help)
% (This matches ConTeXt defaults)
%%% ToC (table of contents) APPEARANCE
% --------------------------------------------------------------------
\usepackage[nottoc,notlof,notlot]{tocbibind} % Put the bibliography in the ToC
\usepackage[titles]{tocloft} % Alter the style of the Table of Contents
\renewcommand{\cftsecfont}{\rmfamily\mdseries\upshape}
\renewcommand{\cftsecpagefont}{\rmfamily\mdseries\upshape} % No bold!
\usepackage{setspace}
\onehalfspacing
%%% END Article customizations

\renewcommand{\sectionmark}[1]{\markright{\ #1}}

%%% The "real" document content comes below...
\begin{document}
\newgeometry{margin=2.5cm}
%\begin{titlepage}
\thispagestyle{empty}
\newcommand{\HRule}{\rule{\linewidth}{0.6mm}} % Defines a new command for the horizontal lines, change thickness here
%\center % Center everything on the page
 
%----------------------------------------------------------------------------------------
%	TITLE SECTION
%----------------------------------------------------------------------------------------

%\begin{spacing}{1.5}
%{\vspace{-4cm} \Large Flexible Instrumental Variables Distributional Regression:}\\[.35cm]
%{\Large A Simulation and Empirical Study}\\% Title of your document
%\end{spacing}
%\HRule \\[4cm]

%----------------------------------------------------------------------------------------
%	HEADING SECTIONS
%----------------------------------------------------------------------------------------
%\textsc{Georg-August Universität G\"ottingen}\\[1.5cm] % Name of your university/college
%\textsc{Master of Science's Thesis}\\[0.5cm] % Major heading such as course name

%\text{\large Master Thesis presented to the Department of Economics at the}\\[.3cm]
%\text{\large Georg-August-University G\"ottingen} \\[.3cm]
%\text{\large with a working time of 20 weeks}\\[3cm]

%\text{\large In partial fulfillment of the requirements for the degree}\\[.3cm]
%\text{\large Master of Science (M.Sc.)}\\[4cm]

%----------------------------------------------------------------------------------------
%	AUTHOR SECTION
%----------------------------------------------------------------------------------------
%\begin{minipage}{0.4\textwidth}
%\begin{flushleft} \large
%Author:\\
%Guillermo Briseño Sanchez\\
%Student ID: 21565589
%\end{flushleft}
%\end{minipage}
%~
%\begin{minipage}{0.4\textwidth}
%\begin{flushright} \large
%Supervisors: \\
%Prof. Dr. Thomas Kneib\\ % Supervisor's Name
%Maike Hohberg, M.A.
%\end{flushright}
%\end{minipage}\\[1cm]

%\text{\large Guillermo Briseño Sanchez}\\
%\text{from: Nuevo Laredo, Mexico}\\
%\text{(21565589)}\\[2cm]

%\text{Supervised by:}\\
%\text{Prof. Dr. Thomas Kneib}\\
%\text{Maike Hohberg, M.A.}\\[.7cm]

%----------------------------------------------------------------------------------------
%	DATE SECTION
%----------------------------------------------------------------------------------------
%{Submitted: February 2, 2018}\\[3.2cm] % Date, change the \today to a set date if you want to be %precise
%----------------------------------------------------------------------------------------
%	LOGO SECTION
%----------------------------------------------------------------------------------------
%\includegraphics{figures/logo}\\[1cm]
 
%----------------------------------------------------------------------------------------
%\vfill % Fill the rest of the page with whitespace
%\end{titlepage}
%\restoregeometry


\thispagestyle{empty}
\begin{center}
    \large
    \textbf{This is a change....}
    \vspace{0.03cm}
    
    \large    
    \vspace{0.03cm}
    %\textbf{Abstract}
\end{center}




\pagenumbering{arabic}
%\pagestyle{fancy}
%\fancyhf{}
%\rhead{\nouppercase{\rightmark}}
%\lhead{G. Briseño}
%\rfoot{\thepage}
%\renewcommand{\headrulewidth}{1pt}
%\renewcommand{\footrulewidth}{1pt}



%%%%%%%%%%%%%%%%%%%%%%%%%%%%%%%%%%%%%%%%%%%%%%%%%%%%%%%%%%%
%%%%%%%%%%%%%%%%%%%%%%%%%%%%%%%%%%%%%%%%%%%%%%%%%%%%%%%%%%%
%%%%%%%%%%%%%%%%%%%%%%%%%%%%%%%%%%%%%%%%%%%%%%%%%%%%%%%%%%%%%%%%%%%%%%%%%%%%%%%%%%%%%%%%%%%%%%%%%%%%%%%%%%%%%%%%%%%%%%
%%%%%%%%%%%%%%%%%%%%%%%%%%%%%%%%%%%%%%%%%%%%%%%%%%%%%%%%%%%
%%%%%%%%%%%%%%%%%%%%%%%%%%%%%%%%%%%%%%%%%%%%%%%%%%%%%%%%%%%%%%%%%%%%%%%%%%%%%%%%%%%%%%%%%%%%%%%%%%%%%%%%%%%%%%%%%%%%%%
%%%%%%%%%%%%%%%%%%%%%%%%%%%%%%%%%%%%%%%%%%%%%%%%%%%%%%%%%%%             2   S   G   A   M   L   S   S
%%%%%%%%%%%%%%%%%%%%%%%%%%%%%%%%%%%%%%%%%%%%%%%%%%%%%%%%%%%%%%%%%%%%%%%%%%%%%%%%%%%%%%%%%%%%%%%%%%%%%%%%%%%%%%%%%%%%%%
%%%%%%%%%%%%%%%%%%%%%%%%%%%%%%%%%%%%%%%%%%%%%%%%%%%%%%%%%%%
%%%%%%%%%%%%%%%%%%%%%%%%%%%%%%%%%%%%%%%%%%%%%%%%%%%%%%%%%%%
\subsection*{Simulation settings}
\underline{True data generating process:}\\
Signal variables:
\begin{center}
\begin{tabular}{cccc}
(instrument) & (endogenous) & (unobserved) & \quad (exogenous) \\
$x_{IV}$ &  $x_{en}$ & $x_{u}$ & $x_{exo1} \quad x_{exo2} \quad x_{exo3}$ 
\end{tabular}
\end{center}

Around 10 noise variables, denoted by: $x^{noise}_s$, $s = 1, \dots, 10$. All noise covariates have the following distribution: $x^{noise}_s \sim \mathcal{N}(1,3^2), \quad \forall$ $s = 1,\dots, 10.$
We will focus on binary endogenous variables, i.e.\ endogenous binary treatments.

Sample $N$ observations of the exogenous regressors, unobserved confounder, and the instrument from independent, univariate Uniform distributions:
\begin{equation*}
    x_{exo_1} \sim \mathcal{U}[0,1], \quad x_{exo_2} \sim \mathcal{U}[0,1], \quad   x_{exo_3} \sim \mathcal{U}[0,1], \quad x_{u} \sim \mathcal{U}[0,1], \quad  x_{IV} \sim \mathcal{U}[0,1] 
    \label{datageneration}
    \end{equation*}
Generate the additive predictor of endogenous regressor: 
    \begin{align*}
    \eta^{en} &= \phi_1 f_d(x_u) + \phi_2 f_d(x_{IV})
    \end{align*}
Obtain the location parameter of the endogenous regressor’s distribution by applying the inverse logit function to the additive predictor $\eta^{en}_{}$. Sample $N$ observations of $x_{en}$ from the desired distribution.
    \begin{align*}
    \vartheta^{en} &= g_1(\eta^{en})^{-1}  = \frac{\exp(\eta^{en})}{1+\exp(\eta^{en})}\\
    x_{en} &\sim Bin(1,\vartheta^{en}_{})
    \end{align*}
Using the sampled binary treatment $x_{en}$, generate the response $y$ from an arbitrary distribution by applying a non-linear function to the covariates, e.g.
    \begin{align*}
    \eta^{y}_{1} &= f_{d}(x_{exo1}) + f_{d}(x_{exo3}) + x_{en}\beta_{en} + f_d(x_u) \\
    \eta^{y}_{2} &= f_{d}(x_{exo2}) + f_{d}(x_{exo3}) +  x_{en}\gamma_{en} + f_d(x_u) 
    \end{align*}
    Obtain the response distribution parameters by applying a suitable response function to each additive predictor $\eta^{y}_{1}$ and $\eta^{y}_{2}$. Sample $N$ observations of $y$ from the desired conditional response distribution
    \begin{align*}
    \vartheta^{y}_{k} &= g_k(\eta^{y}_k)^{-1} \\
    y &\sim f_y(y|\vartheta^{y}_{1}, \dots, \vartheta^{y}_{K})
    \label{responsegeneration}
    \end{align*}
Where $f_d(\bigcdot)$ are some non-linear functions. The parameters $\phi_1, \phi_2$ are used to regulate the strength of the instrument: $\rho(f_d(x_{IV}), x_{en})$, and the severeness of the endogeneity: $\rho(x_{en},f_d(x_{u}))$.  



\pagebreak




%%%%---------------------------------------------------------------------------------
%%%%------------------- S E C T I O N     N O M E N C L A T U R E -------------------

%%% V A R I A B L E S

\nomenclature[C, 04]{$\boldsymbol{y}$}{Response vector.}
\nomenclature[C, 05]{$y$}{Response variable.}
\nomenclature[C, 01]{$\boldsymbol{X}$}{Design matrix of discrete explanatory variables.}
\nomenclature[C, 02]{$\boldsymbol{Z}$}{Design matrix of a continuous explanatory variable.}
\nomenclature[C, 06]{$x_{ex}$}{Exogenous regressor.}
\nomenclature[C, 07]{$x_{en}$}{Endogenous regressor.}
\nomenclature[C, 08]{$x_{IV}$}{Instrumental variable.}
\nomenclature[C, 09]{$x_{u}$}{Unobservable regressor.}
\nomenclature[C, 03]{$\boldsymbol{P}$}{Penalty matrix.}
%\nomenclature[C, 08]{$z_{\bigcdot}$}{Continuous explanatory variable.}


%%% S E T S  &  I N D I C E S 
\nomenclature[A, 02]{$K$}{Set of response distribution parameters: $k = 1,\dots,K$.}
\nomenclature[A, 03]{$M$}{Set of endogenous variable distribution parameters: $m = 1,\dots,M$.}
\nomenclature[A, 01]{$I$}{Set of observations: $i = 1, \dots,n$.}
\nomenclature[A, 04]{$D$}{Set of continuous functions: $d = 1,\dots,D$.}
\nomenclature[A, 07]{$N$}{Sample size.}
\nomenclature[A, 08]{$C$}{Number of Monte Carlo replications: $c = 1, \dots, C$.}
\nomenclature[A, 05]{$J$}{Set of explanatory variables: $j = 1, \dots, J$.}
\nomenclature[A, 06]{$R$}{Set of first-stage regression explanatory variables: $r = 1, \dots, J-1 +$ instrument.}
%% DISTRIBUTIONS
\nomenclature[B]{$\mathcal{U}$}{Uniform distribution with support $[a,b]$.}
\nomenclature[B]{$\mathcal{B}$}{Binomial distribution with parameter $\mu$.}
\nomenclature[B]{$\mathcal{N}$}{Normal distribution with parameters $\mu$, and $\sigma$.}
\nomenclature[B]{$\mathcal{G}$}{Gamma distribution with parameters $\mu$, and $\sigma$.}
\nomenclature[B]{$\mathcal{D}$}{Dagum distribution with parameters $\mu$, $\sigma$, and $\nu$.}

%%%%%%%%%%%%%%
% O T H E R    S Y M B O L S %%%%%%%%%%% 
\nomenclature[E, 02]{$\eta_{\bigcdot}^{\bigcdot}$}{Additive predictor of the $k$-th, or $m$-th distribution parameter.}
\nomenclature[E, 01]{$\boldsymbol{\theta}_{\bigcdot}$}{Vector of $K$ or $M$ distribution parameters:  $\boldsymbol{\theta} = (\mu, \sigma, \nu, \tau)$.  }
\nomenclature[E, 03]{$\vartheta^{y}_{k}$}{$k$-th distribution parameter of the response.}
\nomenclature[E, 04]{$\vartheta^{en}_{m}$}{$m$-th distribution parameter of the endogenous regressor.}
\nomenclature[E, 05]{$\beta_{\bigcdot}$}{Regression coefficient.}
\nomenclature[E, 06]{$\hat{\beta}_{\bigcdot}$}{Estimates of the regression coefficient $\beta$.}
\nomenclature[E, 07]{$\phi$}{Simulation control parameter.}
\nomenclature[E, 08]{$\gamma$}{Penalized smoother coefficient.}
\nomenclature[E, 09]{$\lambda$}{Smoothing parameter. }

%%%%%%%%%%%%%
%% F U N C T I O N S
\nomenclature[D, 04]{$\hat{f}(\bigcdot)$}{Estimated function of a continuous covariate.}
\nomenclature[D, 01]{$f_y(\bigcdot)$}{Response distribution function.} 
\nomenclature[D, 02]{$f_{en}(\bigcdot)$}{Distribution function of the endogenous regressor.}
\nomenclature[D, 03]{$f_d(\bigcdot)$}{The $d$-th continuous function.}
\nomenclature[D, 05]{$g_{\bigcdot}(\bigcdot)$}{$m$-th or $k$-th link function of the $m$-th or $k$-th distribution parameter.}
\nomenclature[D, 06]{$h_{\bigcdot}(\bigcdot)$}{$m$-th or $k$-th response function of the $m$-th or $k$-th distribution parameter.}
\nomenclature[D, 07]{$\ell(\bigcdot)$}{Log-likelihood function.}
\nomenclature[D, 08]{$\ell_p(\bigcdot)$}{Penalized log-likelihood function.}


%%%%---------------------------------------------------------------------------------
%%%%------------------- S E C T I O N     N O M E N C L A T U R E -------------------
\nomenclature[F, 07]{2SDR}{Two-Stage Distributional Regression.}

\pagebreak
%\section*{References}
%\addcontentsline{toc}{section}{References}
\rhead{References}
\bibliography{biblio}

\end{document}
